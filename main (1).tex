\documentclass{article}
\usepackage{graphicx} % Required for inserting images

\title{Research Proposal: determining the most energy efficient method of doing vertical dynamic movement while rock climbing}
\author{Sam Jensen}
\date{February 2026}

\begin{document}

\maketitle

\section{Technical Terms}


\begin{description}
    \item[Dyno:] Short for dynamic movement. It simply refers to the act of "jumping" while climbing. This means all four limbs come off the wall before the hands grab onto another hold.
    \item[COM:] Short for center of mass. Our center of mass is located at the center of the torso.
    \item[Launch Angle:] It is the angle between the feet, and the COM. It determines the angle at which the body moves when the jump is complete.
    
    \item[Finish Hold:] The hold that, when grabbed, marks the end of a movement.
\end{description}


\section{Abstract}
Because the dyno is one of the most common movements in rock climbing, many climbers will benefit by refining their technique to make the movement more energy efficient. When holds are too far to reach, one must jump to reach further. This study will be determining the most energy minimizing way to complete a vertical dyno. While the word dynamic movement is a very broad term, this research focuses on a more precise movement. It is interested in understanding the most efficient way to jump vertically from two hands and two feet, to grab the finish hold with both hands. The hands and feet which the climber jumps from are set up such that they are level to one another.

The dyno movement is comprised of three main parts. Firstly is the depth the COM reaches before the launch is initiated. Secondly is the path the COM travels during the launch. Lastly is the angle at which the COM is propelled as the launch is completed. All three of these factors will influence the velocity, and direction the dyno sends the body. In the ideal scenario the COM is as close to the wall as possible, and as high as possible when the finish hold is caught. This is because when the arms are able to generate the most force, as efficiently as possible, when in the bent position. Should the body be too far from the finish hold, the arms will have to extend to reach it. This puts the arms in a weaker position, requiring more energy in order to keep oneself from falling off the wall. Essentially this study will be optimizing the three previously mentioned factors to create the most energy efficient dyno.

\section{Completed Research}

There have been some several studies completed in this relating to this topic. Kinematic variations between successful and unsuccessful
dynamic rock climbing movements is a paper written by Max Jarvis\cite{Jarvis2021ClimbingDyno}. Jarvis evaluates optimal way for the leg and arm muscles to pull. Another piece of notable research is Biomechanics of the two-handed dyno technique for sport climbing, by Franz Konstantin Fuss and Günther Niegl\cite{Fuss2010Dyno}. This research measures height required to perform successful dynos.

\bibliographystyle{IEEEtran}

\bibliography{references}

\end{document}
